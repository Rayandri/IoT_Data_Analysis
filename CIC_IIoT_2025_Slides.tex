\documentclass[aspectratio=169,11pt]{beamer}

% Theme and colors
\usetheme{Madrid}
\usecolortheme{whale}
\setbeamertemplate{navigation symbols}{}
\setbeamertemplate{footline}[frame number]

% Packages
\usepackage[utf8]{inputenc}
\usepackage[T1]{fontenc}
\usepackage{graphicx}
\usepackage{booktabs}
\usepackage{amsmath}
\usepackage{array}
\usepackage{xcolor}

% Custom colors
\definecolor{epita}{RGB}{0,84,147}
\definecolor{darkblue}{RGB}{26,54,93}
\definecolor{lightgray}{RGB}{245,245,245}

\setbeamercolor{title}{fg=white,bg=epita}
\setbeamercolor{frametitle}{fg=white,bg=epita}
\setbeamercolor{structure}{fg=epita}

% Title information
\title{CIC-IIoT-2025 Security Analysis}
\subtitle{Machine Learning for Intrusion Detection in Industrial IoT}
\author{Yahya Ahachim}
\institute{ML Security -- EPITA SCIA 2026}
\date{January 2025}

\begin{document}

%==============================================================================
% Title slide
%==============================================================================
\begin{frame}
\titlepage
\end{frame}

%==============================================================================
% Agenda
%==============================================================================
\begin{frame}{Agenda}
\begin{enumerate}
    \item Dataset Overview
    \item Data Exploration
    \item Anomaly Detection Methods
    \item Classification Methods
    \item Adversarial Machine Learning
    \item Results Comparison
    \item Conclusions and Recommendations
\end{enumerate}
\end{frame}

%==============================================================================
% Dataset Overview
%==============================================================================
\begin{frame}{Dataset Overview}
\begin{columns}
\begin{column}{0.5\textwidth}
\textbf{CIC-IIoT-2025 Dataset}
\vspace{0.5cm}

\begin{table}
\centering
\begin{tabular}{lr}
\toprule
\textbf{Attribute} & \textbf{Value} \\
\midrule
Total Samples & 227,191 \\
Features & 94 \\
Attack Ratio & 39.79\% \\
Attack Categories & 7 \\
Specific Attacks & 60 \\
\bottomrule
\end{tabular}
\end{table}
\end{column}
\begin{column}{0.5\textwidth}
\textbf{Attack Categories}
\vspace{0.3cm}
\begin{itemize}
    \item Reconnaissance (37.2\%)
    \item DoS (20.4\%)
    \item DDoS (20.0\%)
    \item Man-in-the-Middle (8.9\%)
    \item Malware (8.3\%)
    \item Web Attacks (3.1\%)
    \item Brute Force (2.1\%)
\end{itemize}
\end{column}
\end{columns}
\end{frame}

%==============================================================================
% Attack Distribution
%==============================================================================
\begin{frame}{Attack Distribution}
\begin{center}
\includegraphics[width=0.75\textwidth]{figures/class_distribution.png}
\end{center}
\end{frame}

%==============================================================================
% Key Features
%==============================================================================
\begin{frame}{Key Discriminative Features}
\begin{columns}
\begin{column}{0.5\textwidth}
\begin{table}
\centering
\small
\begin{tabular}{lr}
\toprule
\textbf{Feature} & \textbf{Corr.} \\
\midrule
network\_mss\_max & 0.526 \\
network\_mss\_avg & 0.525 \\
network\_header-length\_min & 0.464 \\
network\_protocols\_dst\_count & 0.423 \\
network\_packets\_all\_count & 0.367 \\
\bottomrule
\end{tabular}
\end{table}
\vspace{0.3cm}
\textbf{Key Insight:} TCP MSS and protocol diversity are strong attack indicators
\end{column}
\begin{column}{0.5\textwidth}
\includegraphics[width=\textwidth]{figures/correlation_heatmap.png}
\end{column}
\end{columns}
\end{frame}

%==============================================================================
% Anomaly Detection Methods
%==============================================================================
\begin{frame}{Anomaly Detection Methods}
\textbf{Unsupervised Learning} -- Trained on benign traffic only
\vspace{0.5cm}

\begin{table}
\centering
\begin{tabular}{p{4cm}p{8cm}}
\toprule
\textbf{Method} & \textbf{Description} \\
\midrule
Isolation Forest & Tree-based isolation of anomalies via random partitioning \\
One-Class SVM & Boundary learning in high-dimensional feature space \\
Local Outlier Factor & Local density deviation detection using k-neighbors \\
\bottomrule
\end{tabular}
\end{table}
\vspace{0.5cm}
\textbf{Objective:} Detect zero-day attacks without requiring labeled attack samples
\end{frame}

%==============================================================================
% Anomaly Detection Results
%==============================================================================
\begin{frame}{Anomaly Detection Results}
\begin{table}
\centering
\begin{tabular}{lccccc}
\toprule
\textbf{Model} & \textbf{Precision} & \textbf{Recall} & \textbf{F1} & \textbf{Bal. Acc.} & \textbf{MCC} \\
\midrule
Isolation Forest & 0.883 & 0.776 & 0.826 & 0.837 & 0.678 \\
One-Class SVM & 0.585 & 0.848 & 0.692 & 0.623 & 0.275 \\
\textbf{LOF} & \textbf{0.887} & \textbf{0.803} & \textbf{0.843} & \textbf{0.851} & \textbf{0.704} \\
\bottomrule
\end{tabular}
\end{table}
\vspace{0.5cm}
\begin{center}
\textbf{Best Performer: Local Outlier Factor}
\end{center}
\end{frame}

%==============================================================================
% Anomaly Detection Comparison Figure
%==============================================================================
\begin{frame}{Anomaly Detection Comparison}
\begin{center}
\includegraphics[width=0.7\textwidth]{figures/anomaly_detection_comparison.png}
\end{center}
\end{frame}

%==============================================================================
% Classification Methods
%==============================================================================
\begin{frame}{Classification Methods}
\textbf{Supervised Learning} -- Using labeled attack and benign samples
\vspace{0.5cm}

\begin{table}
\centering
\begin{tabular}{p{4cm}p{8cm}}
\toprule
\textbf{Method} & \textbf{Description} \\
\midrule
Random Forest & Ensemble of decision trees with majority voting \\
Gradient Boosting & Sequential error correction through boosting \\
SVM (RBF Kernel) & Kernel-based non-linear separation \\
\bottomrule
\end{tabular}
\end{table}
\vspace{0.5cm}
\textbf{Objective:} Accurately classify known attack types with high precision
\end{frame}

%==============================================================================
% Classification Results
%==============================================================================
\begin{frame}{Classification Results}
\begin{table}
\centering
\small
\begin{tabular}{lcccccc}
\toprule
\textbf{Model} & \textbf{Prec.} & \textbf{Recall} & \textbf{F1} & \textbf{MCC} & \textbf{AUPRC} & \textbf{AUC} \\
\midrule
Random Forest & 0.992 & 0.863 & 0.923 & 0.883 & 0.944 & 0.958 \\
\textbf{Gradient Boosting} & \textbf{0.991} & \textbf{0.866} & \textbf{0.925} & \textbf{0.885} & \textbf{0.944} & \textbf{0.959} \\
SVM (RBF) & 0.963 & 0.797 & 0.872 & 0.807 & 0.924 & 0.932 \\
\bottomrule
\end{tabular}
\end{table}
\vspace{0.5cm}
\begin{center}
\textbf{Best Performer: Gradient Boosting}
\end{center}
\end{frame}

%==============================================================================
% Classification Comparison Figure
%==============================================================================
\begin{frame}{Classification Comparison}
\begin{center}
\includegraphics[width=0.7\textwidth]{figures/classification_comparison.png}
\end{center}
\end{frame}

%==============================================================================
% Confusion Matrices
%==============================================================================
\begin{frame}{Confusion Matrices}
\begin{center}
\includegraphics[width=0.8\textwidth]{figures/confusion_matrices.png}
\end{center}
\vspace{0.2cm}
\textbf{Key:} High precision ($>$96\%) minimizes false alarms; Good recall ($>$80\%) detects most attacks
\end{frame}

%==============================================================================
% Adversarial ML Introduction
%==============================================================================
\begin{frame}{Adversarial Machine Learning}
\textbf{Fast Gradient Sign Method (FGSM)}
\vspace{0.3cm}

Adversarial examples crafted using gradient information:
\begin{equation*}
x_{adv} = x + \epsilon \cdot \text{sign}(\nabla_x J(\theta, x, y))
\end{equation*}

\vspace{0.3cm}
\begin{itemize}
    \item White-box attack assuming knowledge of model parameters
    \item Perturbation magnitude controlled by $\epsilon$
    \item Tests model vulnerability to input manipulation
\end{itemize}
\vspace{0.3cm}
\textbf{Objective:} Assess model robustness in adversarial conditions
\end{frame}

%==============================================================================
% FGSM Attack Results
%==============================================================================
\begin{frame}{FGSM Attack on Linear SVM}
\begin{columns}
\begin{column}{0.45\textwidth}
\begin{table}
\centering
\small
\begin{tabular}{ccc}
\toprule
$\epsilon$ & \textbf{Astute Acc.} & \textbf{Attack Rate} \\
\midrule
0.01 & 86.7\% & 13.3\% \\
0.05 & 26.0\% & 74.0\% \\
0.10 & 16.0\% & 84.0\% \\
0.50 & 3.5\% & 96.5\% \\
1.00 & 1.0\% & 99.0\% \\
\bottomrule
\end{tabular}
\end{table}
\vspace{0.3cm}
Linear models are highly vulnerable to gradient-based attacks
\end{column}
\begin{column}{0.55\textwidth}
\includegraphics[width=\textwidth]{figures/fgsm_attack_analysis.png}
\end{column}
\end{columns}
\end{frame}

%==============================================================================
% Adversarial Robustness Comparison
%==============================================================================
\begin{frame}{Adversarial Robustness ($\epsilon=0.5$)}
\begin{table}
\centering
\begin{tabular}{lccc}
\toprule
\textbf{Model} & \textbf{Astute Acc.} & \textbf{Robust Acc.} & \textbf{Robustness} \\
\midrule
Linear SVM & 89.8\% & 3.5\% & 3.9\% \\
Gradient Boosting & 94.5\% & 43.5\% & 46.0\% \\
\textbf{Random Forest} & \textbf{94.3\%} & \textbf{56.7\%} & \textbf{60.1\%} \\
\bottomrule
\end{tabular}
\end{table}
\vspace{0.5cm}
\begin{center}
\textbf{Most Robust: Random Forest}\\
\vspace{0.2cm}
Ensemble methods provide better adversarial resilience
\end{center}
\end{frame}

%==============================================================================
% Robustness Figure
%==============================================================================
\begin{frame}{Model Robustness Comparison}
\begin{center}
\includegraphics[width=0.7\textwidth]{figures/model_robustness.png}
\end{center}
\end{frame}

%==============================================================================
% Best Models Summary
%==============================================================================
\begin{frame}{Summary: Best Models by Task}
\begin{table}
\centering
\large
\begin{tabular}{lll}
\toprule
\textbf{Task} & \textbf{Best Model} & \textbf{Key Metric} \\
\midrule
Zero-day Detection & Local Outlier Factor & F1 = 0.843 \\
Attack Classification & Gradient Boosting & F1 = 0.925 \\
Adversarial Robustness & Random Forest & 60.1\% robust acc. \\
\bottomrule
\end{tabular}
\end{table}
\end{frame}

%==============================================================================
% Multi-Layer Defense
%==============================================================================
\begin{frame}{Multi-Layer Defense Strategy}
\begin{enumerate}
    \item \textbf{Layer 1: Anomaly Detection (LOF)}
    \begin{itemize}
        \item Zero-day attack early warning
        \item Low computational overhead
    \end{itemize}
    \vspace{0.3cm}
    \item \textbf{Layer 2: Classification (Random Forest / Gradient Boosting)}
    \begin{itemize}
        \item Categorize known attack types
        \item High precision for alert prioritization
    \end{itemize}
    \vspace{0.3cm}
    \item \textbf{Layer 3: Adversarial Defense}
    \begin{itemize}
        \item Input validation and sanitization
        \item Ensemble voting for increased confidence
    \end{itemize}
\end{enumerate}
\end{frame}

%==============================================================================
% Key Takeaways
%==============================================================================
\begin{frame}{Key Takeaways}
\begin{itemize}
    \item Machine learning effectively detects IIoT attacks (F1 $>$ 0.92)
    \vspace{0.3cm}
    \item Ensemble methods provide the best adversarial robustness
    \vspace{0.3cm}
    \item Anomaly detection enables zero-day threat identification
    \vspace{0.3cm}
    \item Linear models are vulnerable to gradient-based attacks
    \vspace{0.3cm}
    \item Combining methods provides defense-in-depth
\end{itemize}
\end{frame}

%==============================================================================
% Recommendations
%==============================================================================
\begin{frame}{Recommendations for Production Deployment}
\textbf{Immediate Actions:}
\begin{itemize}
    \item Deploy Random Forest for robustness and interpretability
    \item Add LOF layer for novel attack detection
    \item Establish feature monitoring for drift detection
\end{itemize}
\vspace{0.5cm}
\textbf{Enhanced Security:}
\begin{itemize}
    \item Implement adversarial training
    \item Weekly model retraining with new threat data
    \item Ensemble voting across multiple models
\end{itemize}
\end{frame}

%==============================================================================
% Future Work
%==============================================================================
\begin{frame}{Future Work}
\begin{itemize}
    \item Deep Learning (LSTM, CNN) for temporal pattern recognition
    \vspace{0.3cm}
    \item Federated Learning for distributed IIoT environments
    \vspace{0.3cm}
    \item Attack-specific detection models for improved granularity
    \vspace{0.3cm}
    \item Real-time feature extraction optimization
\end{itemize}
\end{frame}

%==============================================================================
% Demo
%==============================================================================
\begin{frame}{Demo}
\textbf{Live Notebook Demonstration}
\vspace{0.5cm}
\begin{enumerate}
    \item Data loading and exploration
    \item Model training and evaluation
    \item Adversarial attack simulation
    \item Real-time predictions
\end{enumerate}
\end{frame}

%==============================================================================
% Questions
%==============================================================================
\begin{frame}{Questions}
\begin{center}
\vspace{1cm}
{\Large \textbf{Questions?}}
\vspace{1cm}

Yahya Ahachim\\
ML Security -- EPITA SCIA 2026\\
\vspace{0.5cm}
yahya.ahachim@epita.fr
\end{center}
\end{frame}

%==============================================================================
% Thank You
%==============================================================================
\begin{frame}
\begin{center}
\vspace{1cm}
{\Large \textbf{Thank You}}
\vspace{0.5cm}

CIC-IIoT-2025 Security Analysis\\
Machine Learning for Intrusion Detection
\end{center}
\end{frame}

\end{document}
